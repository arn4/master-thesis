\chapter[Derivation of the quadratic activation ODEs]{Derivation of the quadratic activation differential equations}
\label{app:derivation-quadratic-ode}

The scope of this Appendix is to report the computation of the expected values of the
Equations~\eqref{eq:risk_lambda_expval}~and~\eqref{eq:genericODE} when \(\sigma(x) = x^2\).

Due to the linearity of the expected value, we can reduce the expectation on products of \(\dsp\) and \(\lf\).
Let's start with the population risk. The expected value we need can be expanded to 
\[\begin{split}
  \Explfns \left[  \dsp^2  \right]  =&
    \frac{1}{k^2} \sum_{r, s =1}^k \Explf\left[ \act(\lf_{r}^*) \act(\lf_{s}^*)  \right] +
  \\
  & \frac{1}{p^2} \sum_{j, l =1}^k \Explf\left[ \act(\lf_{j}) \act(\lf_{l}) \right]
  \\
  & - \frac{2}{pk}  \sum_{j=1}^p \sum_{r=1}^k \Explf\left[ \act(\lf_{j}) \act(\lf_{r}^*) \right].
\end{split}\]

In the \(\M\) equation we need instead
\[\begin{split}
  \Explfns \left[  \dsp_{j} \lf_{l}  \right]  =&  
  \frac{1}{k} \sum_{r' =1}^{k} \Explf  \left[\act' (\lf_{j}) \lf_l \act( \lf_{r'}^{*}  )  \right]
   \\
   &  -  \frac{1}{p} \sum_{l' = 1}^{p} \Explf \left[ \act' (\lf_{j}) \lf_l   \act ( \lf_{l'}  ) \right] \;, 
  \\
   \Explfns   \left[  \dsp_{j} \lf_{r}^*  \right]  =&    \frac{1}{k} \sum_{r'=1}^{k}   \Explf   \left[\act' (\lf_{j}) \lf_{r}^* \act( \lf_{r'}^{*}  )  \right]
   \\
   &   -   \frac{1}{p} \sum_{l' = 1}^{p} \Explf \left[ \act' (\lf_{j}) \lf_{r}^*   \act ( \lf_{l'}  ) \right]  \;.
\end{split}\]
In addition to the previous terms, in the equation for \(\Q\) we also have to calculate
\[\begin{split}
  \Explfns \left[  \dsp_{j} \dsp_{l}  \right]  =&  
  \frac{1}{k^2} \sum_{ r ,  r' =1}^{k} \Explf  \left[\act' (\lf_{j}) 
  \act' (\lf_{l}) \act (\lf_{r}^*  ) \act ( \lf_{r'}^{*}  )  \right]  \\
  & + \frac{1}{p^2} \sum_{ l ,  l' =1}^{p} \Explf  \left[\act' (\lf_{j}) 
  \act' (\lf_{l}) \act (\lf_{l}  ) \act ( \lf_{l'}  )  \right]  \\ 
  & - \frac{2}{pk} \sum_{ l' =1}^{p} \sum_{ r =1}^{k} \Explf  \left[\act' (\lf_{j}) 
  \act' (\lf_{l}) \act (\lf_{r}^*  ) \act ( \lf_{l'}  )  \right]  \\
  & + \frac{\Delta}{p^2} \sum_{ l ,  l' =1}^{p} \Explf  \left[\act' (\lf_{j}) 
  \act' (\lf_{l})  \right]
\end{split}\]

These expansions are still valid for any generic activation function.
Before specializing in \(\act(x)=x^2\), we introduce a shorthand in the notation.
We will use \[\omega_{\alpha\beta} \coloneqq \left[\bm{\Omega}\right]_{\alpha\beta},\]
where the indices \(\alpha\) and \(\beta\) can discriminate between teacher and student local fields, as well as the numerical index.
%todo: forse spiega meglio come funzionano alpha e beta.
With this consideration, there are only 4 types of expected values to be computed.
Let's write them explicitly, using our activation function
\[\begin{split}
  \Explf  \left[\act' (\lf^\alpha)  \act' (\lf^\beta)  \right] &= 
  4 \Explf  \left[\lf^\alpha \lf^\beta \right] \\
  %
  \Explf  \left[\act(\lf^\alpha)  \act(\lf^\beta)  \right] &= 
  \Explf  \left[(\lf^\alpha)^2 (\lf^\beta)^2 \right] \\
  %
  \Explf  \left[\act'(\lf^\alpha) \lf^\beta \act(\lf^\gamma) \right] &= 
  2\Explf  \left[\lf^\alpha \lf^\beta (\lf^\gamma)^2\right] \\
  %
  \Explf  \left[\act'(\lf^\alpha) \act'(\lf^\beta) \act(\lf^\gamma) \act(\lf^\delta) \right] &= 
  4\Explf  \left[\lf^\alpha \lf^\beta (\lf^\gamma)^2 (\lf^\delta)^2\right] \\
\end{split}\]
We are left with some expected values of polynomials of gaussian variables. 
Since the local fields all have zero mean, these are nothing but \emph{moments} 
of a Gaussian distribution with multiple variables.
The standard result used to calculate these is the \emph{Isserlis' Theorem}, or 
\emph{Wick's probability Theorem} in Physics literature.
\begin{theorem}[Isserlis' Theorem]
  Let \((X_1,\dots,X_n)\) a zero-mean multivariate normal vector,
  then
  \[
    \E\left[X_{i_1}X_{i_2}\cdots X_{i_k}\right] =
      \sum_{p\in P^2_k}\prod_{\{s,t\}\in p} \E\left[X_{i_s}X_{i_t}\right] =
      \sum_{p\in P^2_k}\prod_{\{s,t\}\in p} \Cov\left[X_{i_s},X_{i_t}\right],
  \]
  where the sum is over \(P^2_k\), the set of all possible \emph{pairings} of \([1,k]\).
\end{theorem}
This is a little generalization of the original Theorem, contributed by Withers\cite{withers1985moments}.
We used a \href{https://github.com/arn4/master-thesis/blob/main/analytical-calculations%20/isserlis.py}{Python script} to compute the pairings calculations.
Alternatively, the account can also be made using the generating function of moments.
In this \href{https://github.com/arn4/master-thesis/blob/main/analytical-calculations%20/isserlis.nb}{Mathematica script} we implement this second possibility, 
although the result is then used for the calculation of Appendix~\ref{app:std_sde}.


Using the Theorem, we obtain
\begin{align*}
  4 \Explf  \left[\lf^\alpha \lf^\beta \right] &= 4\omega_{\alpha\beta}\\
  %
  \Explf  \left[(\lf^\alpha)^2 (\lf^\beta)^2 \right] 
  &= \omega_{\alpha\alpha}  \omega_{\beta\beta} + 2 \omega_{\alpha\beta}^2 \\
  %
  2\Explf  \left[\lf^\alpha \lf^\beta (\lf^\gamma)^2\right] 
  &= 2 \omega_{\alpha\beta}\omega_{\gamma\gamma} + 4 \omega_{\alpha\gamma}\omega_{\beta\gamma}\\
  % 
  4\Explf  \left[\lf^\alpha \lf^\beta (\lf^\gamma)^2 (\lf^\delta)^2\right] 
  &= 4\omega_{\alpha\beta}\omega_{\gamma\gamma}\omega_{\delta\delta} + 8\omega_{\alpha\beta}\omega_{\gamma\delta}^2 + 8\omega_{\alpha\gamma}\omega_{\beta\gamma}\omega_{\delta\delta} +\\
  &\quad16\omega_{\alpha\gamma}\omega_{\beta\delta}\omega_{\gamma\delta}+16\omega_{\alpha\delta}\omega_{\beta\gamma}\omega_{\gamma\delta} + 8\omega_{\alpha\delta}\omega_{\beta\delta}\omega_{\gamma\gamma}
\end{align*}
By retracing all steps backward and making the necessary substitutions,
we can arrive at an explicit form of the Equations~\eqref{eq:genericODE}. In order
to obtain a matrix form such as in the Equations~\eqref{eq:quadraticODEs}, 
we have to write in a closed form all the summations appeared during the derivation
and use the fact that \(\Q\) and \(\P\) are symmetric matrices.
\chapter{Mathematical tools}
\section{Stochastic Diffrential Equations}
The aim of this Section is to give a brief introduction on \emph{Stochastic Differential Equations}.
The discussion is not intended to be extremely formal from a mathematical point of view,
but still provide a complete and consistent exposition of the tools used later in the thesis.
For the readers interested in a formal exposition of the concepts expressed below, please refer to \cite{morters2010brownian}.

\subsection{Stochasic Processes}
\begin{Definition}[Stochasic Processes]
  A \emph{stochastich process} is \(\{X_t, t\in T\}\) is a collection of random variable living in the same probability space.
  The set \(T\) is usally considered as the \emph{time domain}, and can be both discrete or continuous.
\end{Definition}

The most fundamental process is the so called \emph{Wiener process}, introduced hereafter.
\begin{Definition}[Wiener Process]
  A continuous stochastich process \(\{W_t\}_{t\ge0}\) is called \emph{Wiener process} if it satisfy these properties:
  \begin{itemize}
    \item \(W_0 = 0\);
    \item The increment is indipendent from past values:
          \[\forall u\ge0: \quad W_{t+u} - W_t  \text{ is indipendent from } W_s \quad \forall s\in[0,t];\]
    \item The increment is distributed as a normal variable with varince equal to time difference:
          \[W_{t+u}-W_t \sim \gauss{(0,u)};\]
    \item The process in continuous in the time:
          \[\] %TODO: find a definition of continuity that suits for us!
  \end{itemize}
\end{Definition}
The importance of this process arise from the fact that it can be used for modelling
most of the stochastich effects that comes from the microscopic action of many agents
external to the modeled system. The integral\footnote{The definition of the ingral of a stochastich process is given in Section \dots} %TODO: insert section
of the Wiener process is often refered as \emph{Brownian motion}; moreover, the Wienere process it's used to model
white noise. Lastly, the Wiener process can be derived as continuous limit of a 
random walk.

\subsection{Integral over a Wiener Process}
Altought a definition of integral for a generic stochastic process is possible,
for this work is interesting only in giving a brief overview of the Wiener process,
because it is the only one needed in the stochasic differential equations of interest.

Let's start from recalling one possible definition of \emph{Riemann Integral} 
of a function \(f\colon[0,T]\to\Real\):
\[
  \int_0^T f{(t)}\dif t \coloneqq \lim_{N\to+\infty} \sum_{k=0}^{N-1}f{(\hat{t}_k)}(t_{k+1}-t_k)
  \quad \text{ with } t_k = \frac{k}{N}T,\,\,\, \hat{t}_k \in [t_k, t_{k+1}].
\]
There exact choices of \(\hat{t}_k\) do not matter, since the limit is converging 
to the same value in any case.

We can now try to extend the Riemann definition to compute the integral of over a Wiener
process. Let \(g:\coloneqq\Real\to\Real\) the function to be integrated. \(g{(t)}\) is a generic (well behaved in some mathematical sense)
random variable depending from time; it's possible to have \(g{(t)} = \tilde{g}{(W_t)}\) or even depending from another stochastic process.
For convinince we will use the convention 
\(\hat{t}_k^\lambda \coloneqq \lambda t_k + (1-\lambda) t_{k+1}\) with \(\lambda\in[0,1]\)\footnote{
  By fixing a value of \(\lambda\) we are essentialy choosing \(\hat{t}_k\) for every 
  interval in the summation. In principle one can think to use different values of \(\lambda\)
  at each different interval, but as it will be clear in a moment, this choice would be
  really impractical and unconvinient.
}. The definition write as 
\[
  \int_0^T g{(t)}\dif W_t{[\lambda]} \coloneqq \lim_{N\to+\infty} \sum_{k=0}^{N-1}g{(\hat{t}_k^\lambda)}(W_{t_{k+1}}-W_{t_k}).
\]

The first observation to do is that the result won't be a number, as when 
integrating a function, but instead is a random variable, since is calculated as
(limit of) sum of random variables.
\begin{Example}[Integral of Wiener process]
  Let compute the expected value of the integral of \(g{(t)}=2W_t\):
  \[\begin{split}
    \E{\left[\int_0^T 2W_t \dif W_t{[\lambda]}\right]}
      &= \E\left[\lim_{N\to+\infty} \sum_{k=0}^{N-1}2W_{\hat{t}_k^\lambda}(W_{t_{k+1}}-W_{t_k})\right] \\
      &= \E\left[\lim_{N\to+\infty} \sum_{k=0}^{N-1}\left(2W_{\hat{t}_k^\lambda}W_{t_{k+1}}-2W_{\hat{t}_k^\lambda}W_{t_k}
                +(W^2_{t_{k+1}} - W^2_{t_{k+1}}) + (W^2_{\hat{t}_k^\lambda}-W^2_{\hat{t}_k^\lambda}) + (W^2_{t_k}-W^2_{t_k})\right)\right] \\
      &= \E\left[\lim_{N\to+\infty} \sum_{k=0}^{N-1}\left(-(W_{t_{k+1}}-W_{\hat{t}_k^\lambda})^2 + (W_{\hat{t}_k^\lambda}-W_{t_{k}})^2 + (W^2_{t_{k+1}} - W^2_{t_{k}}) \right)\right]
  \end{split}\]
  The first two addend in the summation are always negative (positive), thus we can commute
  the expected value and the summation; the third term is a telescopic sum
  \[\begin{split}
    \E{\left[\int_0^T 2W_t \dif W_t{[\lambda]}\right]}
      &= \lim_{N\to+\infty} \sum_{k=0}^{N-1}\E\left[-(W_{t_{k+1}}-W_{\hat{t}_k^\lambda})^2 + (W_{\hat{t}_k^\lambda}-W_{t_{k}}\right] + \E\left[W^2_{T}\right] \\
      &= \lim_{N\to+\infty} \sum_{k=0}^{N-1}\left(-(t_{k+1}-\hat{t}_k^\lambda) + (\hat{t}_k^\lambda-t_{k})^2\right) + T 
      = T + \lim_{N\to+\infty} \sum_{k=0}^{N-1}\left(-t_{k+1}+2\lambda t_k + 2(1-\lambda)t_{k+1}-t_k\right) \\
      &= T + \lim_{N\to+\infty} \sum_{k=0}^{N-1}(1-2\lambda)(t_{k+1}-t_k) = 2(1-\lambda)T
  \end{split}\]
\end{Example}
It is evident that the choice of \(\lambda\) it's important to define the result.
One can choose infinitly many different stochastich integrations, but the two most common
are \emph{Stratonovich Integration} and \emph{Itô Integration}. They lead to different result,
with different general properties. As presented in the folliwing sections, the choice of a spefic
definition of integration has to be made base on the properties that the modelization requires.

\subsubsection{Stratonovich Integration}
The first definition of integration is the one obtained for \(\lambda=\frac12\).
\begin{Definition}[Stratonovich Integration]
  The \emph{Stratonovich Integration} of a function \(g{(t)}\) over Wiener process is
  \[\int_0^T g{(t)}\circ\dif W_t \coloneqq \lim_{N\to+\infty} \sum_{k=0}^{N-1}g{\left(\frac{t_{k}+t_{k+1}}{2}\right)}(W_{t_{k+1}}-W_{t_k}).\]
\end{Definition}

\subsubsection{Itô Integration}
The second deffirent definition of integration is obtained with \(\lambda=1\).
\begin{Definition}[Itô Integration]
  The second definition of integration is obtained by choosing \(\lambda=1\).
  The \emph{Itô Integration} of a function \(g{(t)}\) over Wiener process is
  \[\int_0^T g{(t)}\dif W_t \coloneqq \lim_{N\to+\infty} \sum_{k=0}^{N-1}g{(t_k)}(W_{t_{k+1}}-W_{t_k}).\]
\end{Definition}

\subsection{Stochastich Differential Equations}
The \emph{stochastic differential equation} arise from the need to describe the
presence of intrisic noise that effects the evolution. Let \(X\) a quantity whose
time evolution is regulated by a determistic term \(\mu{(X,t)}\), as well as a noisy
term \(\xi_t\). We would like to write an equation similar to
\[\dod{X}{t} = \mu{(X,t)} \textcolor{red}{+ \sigma{(X,t)}\xi_t},\]
where the noisy term is highlated because we have not a mathematical definition
of it. Of course \(\xi_t\) must be some sort of stochastic process as defined in 
the previous section. This lead to the first important conclusion: the solution of the
equation will not be a function, but a stochastic process too.

Let try to make the things more formal and give a definition for \(\xi_t\). 
We require the following properties to be satisfied:
\begin{enumerate}
  \item the noise is zero-mean \[\E\left[\xi_t\right] = 0 \quad \forall t;\]
  \item the noise is completelly uncorrelated  \[\E\left[\xi_t\xi_{t'}\right] = 0 \quad \forall t\neq t';\]
  \item the noise is stationary (denoting by \(p\) the probability density function)
        \[p{(\xi_{t_1},\dots,\xi_{t_k})} = p{(\xi_{t_1 + t},\dots,\xi_{t_k + t})} \quad \forall t.\]
\end{enumerate}
The generality is not lost with the first property, since the excpected value produces a drift,
that can be included in the function \(\mu\). Also the third property is not restricting too much 
the type of noise, since \(\sigma\) can incorporate eventual variation of the variance\footnote{
  Of course the property 3 is not just asking to have stationary variance, but the stationarity
  involves the whole distribution.
}. The second property is saying that the noise has no memory of what happened in the past;
the memoryless noise is of course just an approximation of real physics system, but it is still
good enought to model many systems. Moreover, the noise considered in this work is artificially 
generated and it fullsills this property, so the general case is not treated here.







\subsection{Simulation}
\texttt{\href{https://en.wikipedia.org/wiki/Euler–Maruyama_method}}

\subsection{References}
The main source used for the writing of this part are \dots